%! TeX root = ../analysis-main.tex
\documentclass[../analysis-main.tex]{subfiles}

\begin{document}

\problem{2}{Let $S = [0,1] \times [0,1]$ and consider the space $C(S)$ of continuous complex-valued functions on $S$ equipped with the $\sup$-norm.
Define $F \subset C(S)$ by \[ F = \{ f \in C(S) \, : \, \exists \, n \geq 1 \text{ and } g_1, \ldots, g_n \, , \, h_1, \ldots, h_n \in C([0,1]) \text{ such that } f(x, y) = \sum_{k = 1}^n g_k(x) h_k(y) \}. \]
Show that $F$ is dense in $C(S)$.}

\begin{solution}{Esha Datta, James Hughes, Edgar Jaramillo Rodriguez, Jeonghoon Kim, Van Vinh Nguyen, Qianhui Wan}
        Let $S$ and $F$ be given as above.
        In order to use the Stone-Weierstrass theorem, we must show that $F$ is an algebra that is nonvanishing, separates points, and for any $f\in F$, we must also have $\overline{f}\in F$.

        We first note that $F$ is nonvanishing because the constant function $\chi_S\in F$. 

        Next, take any $(x_1,y_1), (x_2,y_2)\in S$.
        Without loss of generality, suppose that $x_1\neq x_2$.
        Let $\epsilon=d(x_1,x_2)$ and denote $A=B_\epsilon/2(x_1)$ and $B=S\backslash A$.
        Then there exists a continuous Urysohn function  $$\rho(x,y)=\frac{d(x, A)}{d(x,A)+d(x,B)}\in F$$ such that $\rho(x_1,y)=1$ and $\rho(x_2,y)=0$.
        Thus, $F$ separates points.

        Now if we examine $f(x,y)=\sum_{k=1}^n g_k(x)h_k(y)\in F,$ we can see that $\overline{f}(x,y)=\overline{\sum_{k=1}^n g_k(x)h_k(y)}=\sum_{k=1}^n \overline{g_k}(x)\overline{h_k}(y)\in F$ since $g_k(x)\in C([0,1])$ implies $\overline{g_k}(x)\in C([0,1])$.

        Finally, $F$ is an algebra because $\left(\sum_{k=1}^n g_k(x)h_k(y)\right) \left(\sum_{j=1}^m g_j(x)h_j(y)\right)$ distributes, giving us another element of $F$.

        Thus, by the Stone-Weierstrass theorem, $F$ is dense in $C(S)$.
\end{solution}


\end{document}

