%! TeX root = ../analysis-main.tex

\problem{5}{
        Show that if $X$ is a separable Hilbert space with orthonormal basis $\{f_i\}_{i \geq 1}$ and $T \in B(X)$ is defined by $T(f_k) = \frac{1}{k} f_{k + 1}$ then $T$ is compact and has no eigenvalues.
}

\begin{solution}{Esha Datta, James Hughes, Edgar Jaramillo Rodriguez, Jeonghoon Kim, Van Vinh Nguyen, Qianhui Wan}
        Recall that an operator on a Hilbert space is called Hilbert-Schmidt if there exists an orthonormal basis $e_n$ such that:
        \begin{equation*}
            \sum_{n=1}^\infty \lVert Ae_n \rVert^2 < \infty.
        \end{equation*}
        Observe, 
        \begin{equation*}
            \sum_{n=1}^\infty \lVert Af_n \rVert^2 = 
            \sum_{n=1}^\infty \frac{1}{k^2} < \infty,
        \end{equation*}
        so $A$ is Hilbert-Schmidt.
        It is a theorem that any Hilbert-Schmidt operator is compact, so $A$ is compact.
        To see that $T$ has no eigenvalues, suppose $Tv = \lambda v$ for some $v\in X$ and $\lambda \in \complex$.
        Since $\{f_n: n\in\naturals\}$ is an orthonormal basis we may express $v$ as 
        \begin{equation*}
            v = \sum_{n=1}^\infty \langle v, f_n\rangle f_n
        \end{equation*}
        So we have that:
        \begin{equation*}
            Tv = \sum_{n=1}^\infty \langle v, f_n\rangle Tf_n = \sum_{n=1}^\infty \frac{1}{n} \langle v, f_n\rangle f_{n+1} = \sum_{n=1}^\infty \lambda \langle v, f_n\rangle f_n.
        \end{equation*}
        Therefore, for all $n\in \naturals$ we require that: $(1/n)\langle v, f_n\rangle = \lambda \langle v, f_{n+1} \rangle$.
        If $\langle v , f_1 \rangle \neq 0$ then the series
        \begin{equation*}
            \sum_{n=1}^\infty |\langle v,f_n\rangle|^2 
        \end{equation*}
        does not converge, which is a contradiction.
        Hence it must be the case that $\langle v , f_1 \rangle = 0 \implies \langle v , f_n \rangle = 0$ for all $n\in\naturals$, so $v=0$.
        Thus $T$ has no eigenvalues.
\end{solution}

