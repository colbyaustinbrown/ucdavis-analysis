%! TeX root = ../analysis-main.tex

\problem{4}{
        Let $H$ be a separable infinite dimensional Hilbert space and suppose that $e_1, e_2, \ldots$ is an orthonormal system in $H$.
        Let $f_1, f_2, \ldots$ be another orthonormal system which is complete (i.e., the closure of the span of $\{f_i\}_i$ is all of $H$).
        Prove that if $\sum_{n = 1}^\infty \lVert e_n - f_n \rVert^2 < 1$ then $\{e_1\}_i$ is also a complete orthonormal system.
}

\begin{solution}{Esha Datta, James Hughes, Edgar Jaramillo Rodriguez, Jeonghoon Kim, Van Vinh Nguyen, Qianhui Wan}
        Let $v \in \overline{Span \{e_n\}_{n=1}^\infty}$, it suffices to show that $v =0$.
        Note that $\langle v, e_n \rangle = 0$ for all $n\in \naturals$.
        Also, because $\{f_n\}$ is a complete orthonormal system, we have from Parseval's Identity that:
        \begin{equation*}
                \lVert v \rVert^2 = \sum_{n=1}^\infty \langle v,f_n \rangle = \sum_{n=1}^\infty \langle v,f_n-e_n \rangle^2
                \leq \sum_{n=1}^\infty \lVert v \rVert^2 \lVert e_n-f_n \rVert^2 
                =\lVert v \rVert^2  \sum_{n=1}^\infty \lVert e_n-f_n \rVert^2,
        \end{equation*}
        where the inequality comes from applying Cauchy-Schwarz.
        Now if $\lVert v \rVert$ were not zero, then dividing both sides by $\lVert v \rVert^2$ and using the given inequality would give us that $1< 1$, which is absurd.
        So it must be the case that $\lVert v \rVert= 0 \implies v=0$, as desired.
\end{solution}

